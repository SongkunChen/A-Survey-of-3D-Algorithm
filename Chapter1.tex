% Activate the following line by filling in the right side. If for example the name of the root file is Main.tex, write
% "...root = Main.tex" if the chapter file is in the same directory, and "...root = ../Main.tex" if the chapter is in a subdirectory.
 
%!TEX root =  A Survey of 3D Algorithm

\chapter[Preface]{A Narrow Introduction to 3D world}
% 这是开始尝试用程序员的方式写工作记录的第一天。2023-11-06
人眼采集视觉信息对物体进行描述,可以从颜色/亮度/大小/等不同维度展开,这些因素会被相机以怎样的方式采集和呈现呢?\par
当人获取2D/3D相机的信息后,又需要以怎样的方式对像素数据加以处理,才能获得人能理解和接受的数据呢?
要解答这些问题,就必须要理解3D信息是怎样被照相机获取的,Depth相机相较于常规的RGB相机,又增添了怎样的处理流程?从3D高维数据降低到2D维度时,哪些信息会有丢失或损毁?\par
为了回答这些问题,就需要对相机成像的过程进行深入了解,知道相机内参/外参/畸变参数的物理含义,知道相机畸变校正/极线校正的基本过程,了解2D图像和空间点云的对应关系。这些内容,在后续这些章节中,会逐一展开详细叙述。

\section{Introduction}

\section{2D Camera Coordinates}
\section{3D World Coordinates}
\section{Camera Distortions and Calibration}


